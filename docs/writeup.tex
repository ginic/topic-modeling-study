%
% File acl2020.tex
%
%% Based on the style files for ACL 2020, which were
%% Based on the style files for ACL 2018, NAACL 2018/19, which were
%% Based on the style files for ACL-2015, with some improvements
%%  taken from the NAACL-2016 style
%% Based on the style files for ACL-2014, which were, in turn,
%% based on ACL-2013, ACL-2012, ACL-2011, ACL-2010, ACL-IJCNLP-2009,
%% EACL-2009, IJCNLP-2008...
%% Based on the style files for EACL 2006 by
%%e.agirre@ehu.es or Sergi.Balari@uab.es
%% and that of ACL 08 by Joakim Nivre and Noah Smith

\documentclass[11pt,a4paper]{article}
\usepackage[hyperref]{acl}
\usepackage{times}
\usepackage{latexsym}
\renewcommand{\UrlFont}{\ttfamily\small}

% This is not strictly necessary, and may be commented out,
% but it will improve the layout of the manuscript,
% and will typically save some space.
\usepackage{microtype}

%\aclfinalcopy % Uncomment this line for the final submission
%\def\aclpaperid{***} %  Enter the acl Paper ID here

%\setlength\titlebox{5cm}
% You can expand the titlebox if you need extra space
% to show all the authors. Please do not make the titlebox
% smaller than 5cm (the original size); we will check this
% in the camera-ready version and ask you to change it back.

\newcommand\BibTeX{B\textsc{ib}\TeX}

\title{Give me a title!}

\author{Virginia Partridge \\
  University of Massachusetts Amherst\\
  \texttt{vcpartridge@umass.edu}
}
\date{}

\begin{document}
\maketitle
\begin{abstract}
TODO
\end{abstract}

\section{Motivation}
Latent Dirichlet Analysis is a widely adopted approach for unsupervised topic modeling on text collections.
\section{Related Work}

\section{Methods}
\subsection{Framework for discussing morphological complexity}
lemmas vs lexemes vs types, paradigms
tokens vs surface forms
slots
Russian as a flective langauage
inflectional vs derivational morphology

\subsection{Latent Dirichlet Analysis}
symmetric vs asymmetric prior
mallet gibbs sampling implementation

\subsection{Evaluation metrics}

\section{Corpus}
\subsection{Conflation methods and vocabulary reduction}




\end{document}